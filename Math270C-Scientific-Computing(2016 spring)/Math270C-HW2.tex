\documentclass[12pt,a4paper]{article}
	%[fleqn] %%% --to make all equation left-algned--

\usepackage[top=1.2in, bottom=1.2in, left=0.7in, right=0.7in]{geometry}
%\usepackage{fullpage}

\usepackage{fancyhdr}\pagestyle{fancy}\rhead{Stephanie Wang}\lhead{Math270C - Homework 2}

\usepackage{amsmath,amssymb,amsthm,amsfonts,microtype,stmaryrd}
%{mathtools,wasysym,yhmath}

\usepackage[usenames,dvipsnames]{xcolor}\newcommand{\blue}[1]{\textcolor{blue}{#1}}\newcommand{\red}[1]{\textcolor{red}{#1}}\newcommand{\gray}[1]{\textcolor{gray}{#1}}
\newcommand{\fgreen}[1]{\textcolor{ForestGreen}{#1}}

\usepackage{mdframed}
	%\newtheorem{mdexample}{Example}
	%\definecolor{warmgreen}{rgb}{0.8,0.9,0.85}
	% --Example:
	% \begin{center}
	% \begin{minipage}{0.8,0.9,0.85\textwidth}
	% \begin{mdframed}[backgroundcolor=warmgreen, 
	% skipabove=4pt,skipbelow=4pt,hidealllines=true, 
	% topline=false,leftline=false,middlelinewidth=10pt, 
	% roundcorner=10pt] 
	%%%% --CONTENTS-- %%%%
	% \end{mdframed}\end{minipage}\end{center}	

\usepackage{graphicx}
\graphicspath{ {/Users/KoraJr/Documents/MATLAB} }
	% --Example:
	% \includegraphics[scale=0.5]{picture name}
%\usepackage{caption} %%% --some awful package to make caption...

%\usepackage{hyperref}\hypersetup{linktocpage,colorlinks}\hypersetup{citecolor=black,filecolor=black,linkcolor=black,urlcolor=black}

%%% --Text Fonts
%\usepackage{times} %%% --Times New Roman for LaTeX
%\usepackage{fontspec}\setmainfont{Times New Roman} %%% --Times New Roman; XeLaTeX only

%%% --Math Fonts
%\renewcommand{\mbf}[1]{\mathbf{#1}} %%% --vector
%\newcommand{\ca}[1]{\mathcal{#1}} %%% --"bigO"
%\newcommand{\bb}[1]{\mathbb{#1}} %%% --"Natural, Real numbers"
%\newcommand{\rom}[1]{\romannumeral{#1}} %%% --Roman numbers

%%% --Quick Arrows
\newcommand{\ra}[1]{\ifnum #1=1\rightarrow\fi\ifnum #1=2\Rightarrow\fi}

\newcommand{\la}[1]{\ifnum #1=1 \leftarrow\fi}

%%% --Special Editor Config
\renewcommand{\ni}{\noindent}
\newcommand{\onum}[1]{\raisebox{.5pt}{\textcircled{\raisebox{-1pt} {#1}}}}
\newcommand{\bbu}{\blacktriangleright}
\newcommand{\wbu}{\vartriangleright}

\newcommand{\claim}[1]{\underline{``{#1}":}}
\newcommand{\prob}[1]{\bf {#1}}

\newcommand{\bgfl}{\begin{flalign*}}
\newcommand{\bga}{\begin{align*}}
\def\beq{\begin{equation}} \def\eeq{\end{equation}}

\renewcommand{\l}{\left}\renewcommand{\r}{\right}

\newcommand{\casebrak}[2]{\left \{ \begin{array}{l} {#1}\\{#2} \end{array} \right.}
%\newcommand{\ttm}[4]{\l[\begin{array}{cc}{#1}&{#2}\\{#3}&{#4}\end{array}\r]} %two-by-two-matrix
%\newcommand{\tv}[2]{\l[\begin{array}{c}{#1}\\{#2}\end{array}\r]}

\newcommand{\dps}{\displaystyle}

\let\italiccorrection=\/
\def\/{\ifmmode\expandafter\frac\else\italiccorrection\fi}


%%% --General Math Symbols
\newcommand{\bc}{\because}
\newcommand{\tf}{\therefore}
\newcommand{\SUM}[2]{\sum\limits_{#1}^{#2}}
\newcommand{\PROD}[2]{\prod\limits_{#1}^{#2}}
\newcommand{\CUP}[2]{\bigcup\limits_{#1}^{#2}}
\newcommand{\CAP}[2]{\bigcap\limits_{#1}^{#2}}
\newcommand{\SUP}[1]{\sup\limits_{#1}}

\renewcommand{\o}{\circ}
\newcommand{\x}{\times}
\newcommand{\ox}{\otimes}

%%% --Special Math Characters
\newcommand{\R}{\mathbb R}%Real number
\newcommand{\N}{\mathbb N}%Nature number
\newcommand{\Z}{\mathbb Z}
\newcommand{\C}{\mathbb C}
\newcommand{\F}{\mathbb F}
\renewcommand{\O}{\mathcal{O}}
\newcommand{\A}{\mathcal{A}}%measurable sets
\renewcommand{\P}{\mathcal{P}}%power set

%%% --REAL ANALYSIS Symbols
\newcommand{\INT}[2]{\int_{#1}^{#2}}
\newcommand{\pdiff}[2]{\frac{\partial{#1}}{\partial{#2}}}
\newcommand{\UPINT}{\bar\int}
\newcommand{\UPINTRd}{\overline{\int_{\bb R ^d}}}
\newcommand{\supp}{\text{supp}}

\newcommand{\leb}{\lambda^\ast} %%% --Lebesgue
\renewcommand{\H}[1]{{\cal H}^{#1}} %%% --Hausdorff
\newcommand{\B}{\mathcal{B}} %%% --Borel set
\newcommand{\cL}{\mathcal{L}}
\newcommand{\I}{\mathcal{I}} %%% --index set
\newcommand{\Supp}[1]{\text{Supp}\left({#1}\right)}

\def\Xint#1{\mathchoice
{\XXint\displaystyle\textstyle{#1}}%
{\XXint\textstyle\scriptstyle{#1}}%
{\XXint\scriptstyle\scriptscriptstyle{#1}}%
{\XXint\scriptscriptstyle\scriptscriptstyle{#1}}%
\!\int}
\def\XXint#1#2#3{{\setbox0=\hbox{$#1{#2#3}{\int}$ }
\vcenter{\hbox{$#2#3$ }}\kern-.6\wd0}}
\def\ddashint{\Xint=}
\def\dashint{\Xint-}

\def\vx{\mathbf x}

%%%%%%%%%%%%%%%%%%%%%%%%%%%%%%%%%%%%%%%%%%%%%%%%%%%%%%%%%%%%%%%%%%%%%%%%%%%%%%%%%%%%%%%%%%%%%%%%%%%%%%%%%%%%%%%%%%%%%%%%%%%%%%%%%%%%%%%%%%%%%%%%%%%%%%%%%%%%%%%%%%%%%%%%%%%%%%%%%%%%%%%%%%%%%%%%%%%%%%
\begin{document}
\subsection*{[T1]} 
\claim{if} Assume $\rho(A) = |\lambda_{max}| < 1$. Since $\|N\|_F$ is a fixed number, we could find $\mu \geq 0$ large enough such that 
$$\rho(A) + \frac{\|N\|_F}{1+\mu} < 1$$
with this we shall get 
$$\|A^k\|_2 \leq (1+\mu)^{n-1} \l(\rho(A) + \frac{\|N\|_F}{1+\mu} \r)^k \ra1 0 \text{ as } k \ra1 \infty$$
\claim{and only if} Suppose $\rho(A) \geq 1$ for contradiction; take $v$ be the corresponding eigenvector to the eigenvalue $\lambda_{max}$, then
$$\|A^k v \|_2 = |\lambda_{max}|^k\|v\|_2 \geq \|v\|_2 > 0$$
this shall contradict the condition $\|A^k\|_2 \ra1 0$ as $k\ra1\infty$. 


\subsection*{[T2]} (a) Suppose $A = M-N$ is singular; take $x\in N(A)$, then
$$Ax = Mx - Nx = 0$$
and $Mx = Nx$. Assuming $M$ is non-singular, apply the inverse on both side we see
$$x = M^{-1}Mx = M^{-1}Nx$$
namely $x$ is an eigenvector of $M^{-1}N$ with eigenvalue $1$, contradicts the condition $\rho(M^{-1}N) < 1$. \\
\\
(b) No. Consider the absurd example: $A = I - I = O$ is the zero matrix, written as the difference of two identity matrices, $b = 0$ is the zero vector. In this case with any given initial guess $x^0$, the iteration gives
$$x^{k+1} = I^{-1}\l(Ix^{k} + 0\r) = x^{k} = \cdots = x^0$$
Nothing diverges. 

\subsection*{[T3]}
Suppose for contradiction that $\rho(M^{-1}N) \geq 1$; take take $v$ be the corresponding eigenvector to the eigenvalue $\lambda_{max}$ (which is the largest eigenvalue in magnitude). Set $b = 0$ and $x^0 = v$, then the iteration gives
\bga
x^1 & = M^{-1}N x^0 = \lambda_{max} v\\
& \vdots \\
x^{k+1} & = M^{-1}Nx^k = \lambda_{max}^{k+1}v
\end{align*}
And unless $\lambda_{max} = 1$, this will be a never converging iteration given $|\lambda_{max}| \geq 1$. In the case of $\lambda_{max} = 1$, just consider $b = Mv$, then with the same initial guess $x^0 = v$, we have 
\bga
x^1 & = M^{-1}N x^0 + M^{-1}Mv = 2v \\
& \vdots\\
x^{k+1} & = M^{-1}N x^k + v = (k+2)v
\end{align*}
and this shall diverge and contradicts the assumption. 

\subsection*{[T4]} The iteration can be formulated as 
$$x^{k+1} = I^{-1}(I - \alpha A)x^k + \alpha b$$
and the iterative matrix is $I - \alpha A$. According to [T3] (with $b$ replaced by $\alpha b$ which is still a constant vector), if $\rho(M^{-1}N) \geq 1$ then there exist scenarios such that the iteration diverges. Now suppose $A$ has both positive and negative eigenvalues, denoted by $\lambda^+, \lambda^-$, then
$$1 - \alpha \lambda^+, 1-\alpha \lambda^- \in \sigma(I - \alpha A)$$
for any given $\alpha \neq 0$, one of which is greater than $1$; therefore, $\rho(M^{-1}N) > 1$. 
\newpage\subsection*{[C1/2]}
The results from running my program (also attached) are 
\begin{verbatim}
XXXX  Gauss-Jacobi 2D Operator Test Output XXXX 
X-Panel Count : 50
Y-Panel Count : 50
X-Wavenumber  : 1
Y-Wavenumber  : 1
Omega         : 1
Gauss-Jacobi Iterations  = 743
Residual norm (L2)       = 0.000994929
Residual norm (Inf)      = 0.00198206

XXXX  SOR  2D Operator Test Output XXXX 
X-Panel Count : 50
Y-Panel Count : 50
X-Wavenumber  : 1
Y-Wavenumber  : 1
Omega         : 1
SOR Iterations           = 380
Residual norm (L2)       = 0.000995524
Residual norm (Inf)      = 0.00256255

XXXX  Gauss-Jacobi 2D Operator Test Output XXXX 
X-Panel Count : 50
Y-Panel Count : 50
X-Wavenumber  : 24
Y-Wavenumber  : 24
Omega         : 1
Gauss-Jacobi Iterations  = 6
Residual norm (L2)       = 0.000803458
Residual norm (Inf)      = 0.00952229

XXXX  SOR  2D Operator Test Output XXXX 
X-Panel Count : 50
Y-Panel Count : 50
X-Wavenumber  : 24
Y-Wavenumber  : 24
Omega         : 1
SOR Iterations           = 8
Residual norm (L2)       = 0.000666814
Residual norm (Inf)      = 0.00286167
\end{verbatim}
\subsection*{[C3]} 
Some other results from running my program are
\\
\begin{center}\red{\large $\bbu$ increasing mesh size $\blacktriangleleft$}\end{center}
\begin{verbatim}
XXXX  Gauss-Jacobi 2D Operator Test Output XXXX 
X-Panel Count : 100
Y-Panel Count : 100
X-Wavenumber  : 1
Y-Wavenumber  : 1
Omega         : 1
Gauss-Jacobi Iterations  = 2980
Residual norm (L2)       = 0.000998181
Residual norm (Inf)      = 0.00199636

XXXX  SOR  2D Operator Test Output XXXX 
X-Panel Count : 100
Y-Panel Count : 100
X-Wavenumber  : 1
Y-Wavenumber  : 1
Omega         : 1
SOR Iterations           = 1499
Residual norm (L2)       = 0.00099629
Residual norm (Inf)      = 0.00234244

XXXX  Gauss-Jacobi 2D Operator Test Output XXXX 
X-Panel Count : 200
Y-Panel Count : 200
X-Wavenumber  : 1
Y-Wavenumber  : 1
Omega         : 1
Gauss-Jacobi Iterations  = 11926
Residual norm (L2)       = 0.000999976
Residual norm (Inf)      = 0.00199995

XXXX  SOR  2D Operator Test Output XXXX 
X-Panel Count : 200
Y-Panel Count : 200
X-Wavenumber  : 1
Y-Wavenumber  : 1
Omega         : 1
SOR Iterations           = 5972
Residual norm (L2)       = 0.000999479
Residual norm (Inf)      = 0.00218814
\end{verbatim}

\newpage\begin{center}\red{\large $\bbu$ varying relaxation parameter $\blacktriangleleft$}\end{center}
\begin{verbatim}
XXXX  Gauss-Jacobi 2D Operator Test Output XXXX 
X-Panel Count : 50
Y-Panel Count : 50
X-Wavenumber  : 1
Y-Wavenumber  : 1
Omega         : 0.25
Gauss-Jacobi Iterations  = 2982
Residual norm (L2)       = 0.000998567
Residual norm (Inf)      = 0.00198929

XXXX  SOR  2D Operator Test Output XXXX 
X-Panel Count : 50
Y-Panel Count : 50
X-Wavenumber  : 1
Y-Wavenumber  : 1
Omega         : 0.25
SOR Iterations           = 2613
Residual norm (L2)       = 0.000999976
Residual norm (Inf)      = 0.00210963

XXXX  Gauss-Jacobi 2D Operator Test Output XXXX 
X-Panel Count : 50
Y-Panel Count : 50
X-Wavenumber  : 1
Y-Wavenumber  : 1
Omega         : 0.5
Gauss-Jacobi Iterations  = 1489
Residual norm (L2)       = 0.000998673
Residual norm (Inf)      = 0.0019895

XXXX  SOR  2D Operator Test Output XXXX 
X-Panel Count : 50
Y-Panel Count : 50
X-Wavenumber  : 1
Y-Wavenumber  : 1
Omega         : 0.5
SOR Iterations           = 1122
Residual norm (L2)       = 0.000998006
Residual norm (Inf)      = 0.00224027

XXXX  Gauss-Jacobi 2D Operator Test Output XXXX 
X-Panel Count : 50
Y-Panel Count : 50
X-Wavenumber  : 1
Y-Wavenumber  : 1
Omega         : 0.75
Gauss-Jacobi Iterations  = 992
Residual norm (L2)       = 0.000994834
Residual norm (Inf)      = 0.00198185

XXXX  SOR  2D Operator Test Output XXXX 
X-Panel Count : 50
Y-Panel Count : 50
X-Wavenumber  : 1
Y-Wavenumber  : 1
Omega         : 0.75
SOR Iterations           = 626
Residual norm (L2)       = 0.000998346
Residual norm (Inf)      = 0.00239897

XXXX  Gauss-Jacobi 2D Operator Test Output XXXX 
X-Panel Count : 50
Y-Panel Count : 50
X-Wavenumber  : 1
Y-Wavenumber  : 1
Omega         : 1
Gauss-Jacobi Iterations  = 743
Residual norm (L2)       = 0.000994929
Residual norm (Inf)      = 0.00198206

XXXX  SOR  2D Operator Test Output XXXX 
X-Panel Count : 50
Y-Panel Count : 50
X-Wavenumber  : 1
Y-Wavenumber  : 1
Omega         : 1
SOR Iterations           = 380
Residual norm (L2)       = 0.000995524
Residual norm (Inf)      = 0.00256255

XXXX  Gauss-Jacobi 2D Operator Test Output XXXX 
X-Panel Count : 50
Y-Panel Count : 50
X-Wavenumber  : 1
Y-Wavenumber  : 1
Omega         : 1.25
Gauss-Jacobi Iterations  = 1816
Residual norm (L2)       = nan
Residual norm (Inf)      = inf

XXXX  SOR  2D Operator Test Output XXXX 
X-Panel Count : 50
Y-Panel Count : 50
X-Wavenumber  : 1
Y-Wavenumber  : 1
Omega         : 1.25
SOR Iterations           = 235
Residual norm (L2)       = 0.00099727
Residual norm (Inf)      = 0.00267221

XXXX  Gauss-Jacobi 2D Operator Test Output XXXX 
X-Panel Count : 50
Y-Panel Count : 50
X-Wavenumber  : 1
Y-Wavenumber  : 1
Omega         : 1.5
Gauss-Jacobi Iterations  = 1055
Residual norm (L2)       = nan
Residual norm (Inf)      = inf

XXXX  SOR  2D Operator Test Output XXXX 
X-Panel Count : 50
Y-Panel Count : 50
X-Wavenumber  : 1
Y-Wavenumber  : 1
Omega         : 1.5
SOR Iterations           = 144
Residual norm (L2)       = 0.000973898
Residual norm (Inf)      = 0.00237577

XXXX  Gauss-Jacobi 2D Operator Test Output XXXX 
X-Panel Count : 50
Y-Panel Count : 50
X-Wavenumber  : 1
Y-Wavenumber  : 1
Omega         : 1.75
Gauss-Jacobi Iterations  = 796
Residual norm (L2)       = nan
Residual norm (Inf)      = inf

XXXX  SOR  2D Operator Test Output XXXX 
X-Panel Count : 50
Y-Panel Count : 50
X-Wavenumber  : 1
Y-Wavenumber  : 1
Omega         : 1.75
SOR Iterations           = 80
Residual norm (L2)       = 0.000958905
Residual norm (Inf)      = 0.002221
\end{verbatim}
We can conclude that increasing mesh size slows down the convergence drastically. On the other hand, Gauss-Jacobi seems to work the best at $\omega = 1$ while SOR can work better with over-relaxation with $\omega = 1.75$. 
\end{document}