\documentclass[12pt,a4paper]{article}
	%[fleqn] %%% --to make all equation left-algned--

% \usepackage[utf8]{inputenc}
% \DeclareUnicodeCharacter{1D12A}{\doublesharp}
% \DeclareUnicodeCharacter{2693}{\anchor}
% \usepackage{dingbat}
% \DeclareRobustCommand\dash\unskip\nobreak\thinspace{\textemdash\allowbreak\thinspace\ignorespaces}
\usepackage[top=2in, bottom=1in, left=1in, right=1in]{geometry}
%\usepackage{fullpage}

\usepackage{fancyhdr}\pagestyle{fancy}\rhead{Stephanie Wang}\lhead{EE236C Project Proposal}
\usepackage{nicefrac}

\usepackage{amsmath,amssymb,amsthm,amsfonts,microtype,stmaryrd}
	%{mathtools,wasysym,yhmath}

\usepackage[usenames,dvipsnames]{xcolor}
\newcommand{\blue}[1]{\textcolor{blue}{#1}}
\newcommand{\red}[1]{\textcolor{red}{#1}}
\newcommand{\gray}[1]{\textcolor{gray}{#1}}
\newcommand{\fgreen}[1]{\textcolor{ForestGreen}{#1}}

\usepackage{mdframed}
	%\newtheorem{mdexample}{Example}
	\definecolor{warmgreen}{rgb}{0.8,0.9,0.85}
	% --Example:
	% \begin{center}
	% \begin{minipage}{0.7\textwidth}
	% \begin{mdframed}[backgroundcolor=warmgreen, 
	% skipabove=4pt,skipbelow=4pt,hidealllines=true, 
	% topline=false,leftline=false,middlelinewidth=10pt, 
	% roundcorner=10pt] 
	%%%% --CONTENTS-- %%%%
	% \end{mdframed}\end{minipage}\end{center}	

\usepackage{graphicx} \graphicspath{{}}
	% --Example:
	% \includegraphics[scale=0.5]{picture name}
%\usepackage{caption} %%% --some awful package to make caption...

\usepackage{hyperref}\hypersetup{linktocpage,colorlinks}\hypersetup{citecolor=black,filecolor=black,linkcolor=black,urlcolor=blue,breaklinks=true}

%%% --Text Fonts
%\usepackage{times} %%% --Times New Roman for LaTeX
%\usepackage{fontspec}\setmainfont{Times New Roman} %%% --Times New Roman; XeLaTeX only

%%% --Math Fonts
\renewcommand{\v}[1]{\ifmmode\mathbf{#1}\fi}
%\renewcommand{\mbf}[1]{\mathbf{#1}} %%% --vector
%\newcommand{\ca}[1]{\mathcal{#1}} %%% --"bigO"
%\newcommand{\bb}[1]{\mathbb{#1}} %%% --"Natural, Real numbers"
%\newcommand{\rom}[1]{\romannumeral{#1}} %%% --Roman numbers

%%% --Quick Arrows
\newcommand{\ra}[1]{\ifnum #1=1\rightarrow\fi\ifnum #1=2\Rightarrow\fi\ifnum #1=3\Rrightarrow\fi\ifnum #1=4\rightrightarrows\fi\ifnum #1=5\rightleftarrows\fi\ifnum #1=6\mapsto\fi\ifnum #1=7\iffalse\fi\fi\ifnum #1=8\twoheadrightarrow\fi\ifnum #1=9\rightharpoonup\fi\ifnum #1=0\rightharpoondown\fi}

%\newcommand{\la}[1]{\ifnum #1=1\leftarrow\fi\ifnum #1=2\Leftarrow\fi\ifnum #1=3\Lleftarrow\fi\ifnum #1=4\leftleftarrows\fi\ifnum #1=5\rightleftarrows\fi\ifnum #1=6\mapsfrom\ifnum #1=7\iffalse\fi\fi\ifnum #1=8\twoheadleftarrow\fi\ifnum #1=9\leftharpoonup\fi\ifnum #1=0\leftharpoondown\fi}

%\newcommand{\ua}[1]{\ifnum #1=1\uparrow\fi\ifnum #1=2\Uparrow\fi}
%\newcommand{\da}[1]{\ifnum #1=1\downarrow\fi\ifnum #1=2\Downarrow\fi}

%%% --Special Editor Config
\renewcommand{\ni}{\noindent}
\newcommand{\onum}[1]{\raisebox{.5pt}{\textcircled{\raisebox{-1pt} {#1}}}}

\newcommand{\claim}[1]{\underline{``{#1}":}}

\renewcommand{\l}{\left}\renewcommand{\r}{\right}

\newcommand{\casebrak}[4]{\left \{ \begin{array}{ll} {#1},&{#2}\\{#3},&{#4} \end{array} \right.}
\newcommand{\ttm}[4]{\l[\begin{array}{cc}{#1}&{#2}\\{#3}&{#4}\end{array}\r]} %two-by-two-matrix
\newcommand{\tv}[2]{\l[\begin{array}{c}{#1}\\{#2}\end{array}\r]}

\def\dps{\displaystyle}

\let\italiccorrection=\/
\def\/{\ifmmode\expandafter\frac\else\italiccorrection\fi}


%%% --General Math Symbols
\def\bc{\because}
\def\tf{\therefore}

%%% --Frequently used OPERATORS shorthand
\newcommand{\INT}[2]{\int_{#1}^{#2}}
% \newcommand{\UPINT}{\bar\int}
% \newcommand{\UPINTRd}{\overline{\int_{\bb R ^d}}}
\newcommand{\SUM}[2]{\sum\limits_{#1}^{#2}}
\newcommand{\PROD}[2]{\prod\limits_{#1}^{#2}}
\newcommand{\CUP}[2]{\bigcup\limits_{#1}^{#2}}
\newcommand{\CAP}[2]{\bigcap\limits_{#1}^{#2}}
% \newcommand{\SUP}[1]{\sup\limits_{#1}}
% \newcommand{\INF}[1]{\inf\limits_{#1}}
\DeclareMathOperator*{\argmin}{arg\,min}
\DeclareMathOperator*{\argmax}{arg\,max}
\newcommand{\pd}[2]{\frac{\partial{#1}}{\partial{#2}}}
\def\tr{\text{tr}}

\renewcommand{\o}{\circ}
\newcommand{\x}{\times}
\newcommand{\ox}{\otimes}

\newcommand\ie{{\it i.e. }}
\newcommand\wrt{{w.r.t. }}
\newcommand\dom{\mathbf{dom\:}}

%%% --Frequently used VARIABLES shorthand
\def\R{\ifmmode\mathbb R\fi}
\def\N{\ifmmode\mathbb N\fi}
\renewcommand{\O}{\mathcal{O}}

\newcommand{\dt}{\Delta t}
\def\vA{\mathbf{A}}
\def\vB{\mathbf{B}}\def\cB{\mathcal{B}}
\def\vC{\mathbf{C}}
\def\vD{\mathbf{D}}
\def\vE{\mathbf{E}}
\def\vF{\mathbf{F}}\def\tvF{\tilde{\mathbf{F}}}
\def\vG{\mathbf{G}}
\def\vH{\mathbf{H}}
\def\vI{\mathbf{I}}\def\cI{\mathcal{I}}
\def\vJ{\mathbf{J}}
\def\vK{\mathbf{K}}
\def\vL{\mathbf{L}}\def\cL{\mathcal{L}}
\def\vM{\mathbf{M}}
\def\vN{\mathbf{N}}\def\cN{\mathcal{N}}
\def\vO{\mathbf{O}}
\def\vP{\mathbf{P}}
\def\vQ{\mathbf{Q}}
\def\vR{\mathbf{R}}
\def\vS{\mathbf{S}}
\def\vT{\mathbf{T}}
\def\vU{\mathbf{U}}
\def\vV{\mathbf{V}}
\def\vW{\mathbf{W}}
\def\vX{\mathbf{X}}
\def\vY{\mathbf{Y}}
\def\vZ{\mathbf{Z}}

\def\va{\mathbf{a}}
\def\vb{\mathbf{b}}
\def\vc{\mathbf{c}}
\def\vd{\mathbf{d}}
\def\ve{\mathbf{e}}
\def\vf{\mathbf{f}}
\def\vg{\mathbf{g}}
\def\vh{\mathbf{h}}
\def\vi{\mathbf{i}}
\def\vj{\mathbf{j}}
\def\vk{\mathbf{k}}
\def\vl{\mathbf{l}}
\def\vm{\mathbf{m}}
\def\vn{\mathbf{n}}
\def\vo{\mathbf{o}}
\def\vp{\mathbf{p}}
\def\vq{\mathbf{q}}
\def\vr{\mathbf{r}}
\def\vs{\mathbf{s}}
\def\vt{\mathbf{t}}
\def\vu{\mathbf{u}}
\def\vv{\mathbf{v}}\def\tvv{\tilde{\mathbf{v}}}
\def\vw{\mathbf{w}}
\def\vx{\mathbf{x}}\def\tvx{\tilde{\mathbf{x}}}
\def\vy{\mathbf{y}}
\def\vz{\mathbf{z}}

\def\diag{\mbox{\textbf{diag}}}

\newcommand{\kp}[1]{{k+{#1}}}
\newcommand{\km}[1]{{k-{#1}}}
\newcommand{\np}[1]{{n+{#1}}}
\newcommand{\nm}[1]{{n-{#1}}}

%%% --Numerical analysis related
%\newcommand{\nxt}{^{n+1}}
%\newcommand{\pvs}{^{n-1}}
%\newcommand{\hfnxt}{^{n+\frac12}}

%%%%%%%%%%%%%%%%%%%%%%%%%%%%%%%%%%%%%%%%%%%%%%%%%%%%%%%%%%%%%%%%%%%%%%%%%%%%%%%%%%%%%%%%%%%%%%%%%%%%%%%%%%%%%%%%%%%%%%%%%%%%%%%%%%%%%%%%%%%%%%%%%%%%%%%%%%%%%%%%%%%%%%%%%%%%%%%%%%%%%%%%%%%%%%%%%%%%%%
\begin{document}
\subsection*{The Optimal Transport Problem}
Let $\Omega \subseteq \R^n$ be a compact set. Given two probability measure $\mu \in \mathcal P(\Omega), \nu \in \mathcal P(\Omega)$, and a cost function $c: \Omega \x \Omega \ra1 \R^+$, the Monge problem \eqref{MP} wishes to find the optimal transport map $T:\Omega \ra0 \Omega$ such that 
\begin{equation} \label{MP} \tag{MP}
T = \argmin\l\{\int c(x, T(x)) d\mu(x) \,\middle\vert\, T\#\mu = \nu \r\}.
\end{equation}
The constraint $T\#\mu = \nu$ can be interpreted as 
$$f \in L^1(\nu) \Longrightarrow f \o T \in L^1(\mu) \mbox{ and } \int f(T(x)) d\mu(x) = \int f(y) d\nu(y).$$
In other words, we look for a transport map to transfer a distribution $\mu$ to another $\nu$, while minimizing the transportation cost $c$. A relaxation to this problem is the Kantarovich problem \eqref{KP} which minimizes over all transport plans:
\begin{equation} \label{KP} \tag{KP}
	\gamma = \argmin\l\{\int c(x, y) d\gamma(x, y) \,\middle\vert\, \pi_x\#\gamma = \mu, \pi_y\#\gamma = \nu \r\}.
\end{equation}
Another equivalent problem is the minimization over solutions of transport equation:
\begin{equation*} %\label{TQ} \tag{TQ}
	(\rho, m) = \argmin \l\{\INT01 \int_{\R^n} \frac{|\vm|^2}{2\rho} dxdt \,\middle\vert\, \partial_t\rho + \nabla\cdot \vm = 0, \vm \cdot \vn_{\partial\Omega} = 0, \rho(0, \cdot) = \mu, \rho(1, \cdot) = \nu \r\}
\end{equation*}
Note that slight abuse of notation: we use $\mu, \nu$ to denote the probability measure or their density function interchangeably here.

\subsection*{Proposed Study}

The proposed project will involve the study of the equivalence of the three problems above, the structure of their constraints, and numerical experiments of applying convex optimization methods to one of the three problems. The theoretical part of this project will emphasize on the studying of the advantages of each of the problems and the physical intuition of the constraints and the minimization objectives. The numerical part of this project will consist of comparison of the potential discrete formulations and their computational costs, implementation (either in C++ or MATLAB) of several convex optimization methods covered in class for the numerically advantageous one out of the three problems, and a performance comparison on a sample problem to which we can obtain an analytical solution.

The theoretical part of this project will be heavily involved with knowledge from functional analysis and partial differential equations. Although these two subjects seem unrelated to our course, the studying of infinite-dimensional minimization problems alone with their discretization could be very intriguing and inspiring.




\end{document}



