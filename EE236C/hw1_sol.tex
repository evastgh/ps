\documentclass[12pt,a4paper]{article}
	%[fleqn] %%% --to make all equation left-algned--

% \usepackage[utf8]{inputenc}
% \DeclareUnicodeCharacter{1D12A}{\doublesharp}
% \DeclareUnicodeCharacter{2693}{\anchor}
% \usepackage{dingbat}
% \DeclareRobustCommand\dash\unskip\nobreak\thinspace{\textemdash\allowbreak\thinspace\ignorespaces}
\usepackage[top=2in, bottom=1in, left=1in, right=1in]{geometry}
%\usepackage{fullpage}

\usepackage{fancyhdr}\pagestyle{fancy}\rhead{Stephanie Wang}\lhead{EE236C homework 1}

\usepackage{amsmath,amssymb,amsthm,amsfonts,microtype,stmaryrd}
	%{mathtools,wasysym,yhmath}

\usepackage[usenames,dvipsnames]{xcolor}
\newcommand{\blue}[1]{\textcolor{blue}{#1}}
\newcommand{\red}[1]{\textcolor{red}{#1}}
\newcommand{\gray}[1]{\textcolor{gray}{#1}}
\newcommand{\fgreen}[1]{\textcolor{ForestGreen}{#1}}

\usepackage{mdframed}
	%\newtheorem{mdexample}{Example}
	\definecolor{warmgreen}{rgb}{0.8,0.9,0.85}
	% --Example:
	% \begin{center}
	% \begin{minipage}{0.7\textwidth}
	% \begin{mdframed}[backgroundcolor=warmgreen, 
	% skipabove=4pt,skipbelow=4pt,hidealllines=true, 
	% topline=false,leftline=false,middlelinewidth=10pt, 
	% roundcorner=10pt] 
	%%%% --CONTENTS-- %%%%
	% \end{mdframed}\end{minipage}\end{center}	

\usepackage{graphicx} \graphicspath{{}}
	% --Example:
	% \includegraphics[scale=0.5]{picture name}
%\usepackage{caption} %%% --some awful package to make caption...

\usepackage{hyperref}\hypersetup{linktocpage,colorlinks}\hypersetup{citecolor=black,filecolor=black,linkcolor=black,urlcolor=blue,breaklinks=true}

%%% --Text Fonts
%\usepackage{times} %%% --Times New Roman for LaTeX
%\usepackage{fontspec}\setmainfont{Times New Roman} %%% --Times New Roman; XeLaTeX only

%%% --Math Fonts
\renewcommand{\v}[1]{\ifmmode\mathbf{#1}\fi}
%\renewcommand{\mbf}[1]{\mathbf{#1}} %%% --vector
%\newcommand{\ca}[1]{\mathcal{#1}} %%% --"bigO"
%\newcommand{\bb}[1]{\mathbb{#1}} %%% --"Natural, Real numbers"
%\newcommand{\rom}[1]{\romannumeral{#1}} %%% --Roman numbers

%%% --Quick Arrows
\newcommand{\ra}[1]{\ifnum #1=1\rightarrow\fi\ifnum #1=2\Rightarrow\fi\ifnum #1=3\Rrightarrow\fi\ifnum #1=4\rightrightarrows\fi\ifnum #1=5\rightleftarrows\fi\ifnum #1=6\mapsto\fi\ifnum #1=7\iffalse\fi\fi\ifnum #1=8\twoheadrightarrow\fi\ifnum #1=9\rightharpoonup\fi\ifnum #1=0\rightharpoondown\fi}

%\newcommand{\la}[1]{\ifnum #1=1\leftarrow\fi\ifnum #1=2\Leftarrow\fi\ifnum #1=3\Lleftarrow\fi\ifnum #1=4\leftleftarrows\fi\ifnum #1=5\rightleftarrows\fi\ifnum #1=6\mapsfrom\ifnum #1=7\iffalse\fi\fi\ifnum #1=8\twoheadleftarrow\fi\ifnum #1=9\leftharpoonup\fi\ifnum #1=0\leftharpoondown\fi}

%\newcommand{\ua}[1]{\ifnum #1=1\uparrow\fi\ifnum #1=2\Uparrow\fi}
%\newcommand{\da}[1]{\ifnum #1=1\downarrow\fi\ifnum #1=2\Downarrow\fi}

%%% --Special Editor Config
\renewcommand{\ni}{\noindent}
\newcommand{\onum}[1]{\raisebox{.5pt}{\textcircled{\raisebox{-1pt} {#1}}}}

\newcommand{\claim}[1]{\underline{``{#1}":}}

\renewcommand{\l}{\left}\renewcommand{\r}{\right}

\newcommand{\casebrak}[4]{\left \{ \begin{array}{ll} {#1},&{#2}\\{#3},&{#4} \end{array} \right.}
%\newcommand{\ttm}[4]{\l[\begin{array}{cc}{#1}&{#2}\\{#3}&{#4}\end{array}\r]} %two-by-two-matrix
%\newcommand{\tv}[2]{\l[\begin{array}{c}{#1}\\{#2}\end{array}\r]}

\def\dps{\displaystyle}

\let\italiccorrection=\/
\def\/{\ifmmode\expandafter\frac\else\italiccorrection\fi}


%%% --General Math Symbols
\def\bc{\because}
\def\tf{\therefore}

%%% --Frequently used OPERATORS shorthand
\newcommand{\INT}[2]{\int_{#1}^{#2}}
% \newcommand{\UPINT}{\bar\int}
% \newcommand{\UPINTRd}{\overline{\int_{\bb R ^d}}}
\newcommand{\SUM}[2]{\sum\limits_{#1}^{#2}}
\newcommand{\PROD}[2]{\prod\limits_{#1}^{#2}}
\newcommand{\CUP}[2]{\bigcup\limits_{#1}^{#2}}
\newcommand{\CAP}[2]{\bigcap\limits_{#1}^{#2}}
% \newcommand{\SUP}[1]{\sup\limits_{#1}}
% \newcommand{\INF}[1]{\inf\limits_{#1}}
\DeclareMathOperator*{\argmin}{arg\,min}
\DeclareMathOperator*{\argmax}{arg\,max}
\newcommand{\pd}[2]{\frac{\partial{#1}}{\partial{#2}}}
\def\tr{\text{tr}}

\renewcommand{\o}{\circ}
\newcommand{\x}{\times}
\newcommand{\ox}{\otimes}

\newcommand\ie{{\it i.e. }}
\newcommand\wrt{{w.r.t. }}
\newcommand\dom{\mathbf{dom\:}}

%%% --Frequently used VARIABLES shorthand
\def\R{\ifmmode\mathbb R\fi}
\def\N{\ifmmode\mathbb N\fi}
\renewcommand{\O}{\mathcal{O}}

\newcommand{\dt}{\Delta t}
\def\vA{\mathbf{A}}
\def\vB{\mathbf{B}}\def\cB{\mathcal{B}}
\def\vC{\mathbf{C}}
\def\vD{\mathbf{D}}
\def\vE{\mathbf{E}}
\def\vF{\mathbf{F}}\def\tvF{\tilde{\mathbf{F}}}
\def\vG{\mathbf{G}}
\def\vH{\mathbf{H}}
\def\vI{\mathbf{I}}\def\cI{\mathcal{I}}
\def\vJ{\mathbf{J}}
\def\vK{\mathbf{K}}
\def\vL{\mathbf{L}}\def\cL{\mathcal{L}}
\def\vM{\mathbf{M}}
\def\vN{\mathbf{N}}\def\cN{\mathcal{N}}
\def\vO{\mathbf{O}}
\def\vP{\mathbf{P}}
\def\vQ{\mathbf{Q}}
\def\vR{\mathbf{R}}
\def\vS{\mathbf{S}}
\def\vT{\mathbf{T}}
\def\vU{\mathbf{U}}
\def\vV{\mathbf{V}}
\def\vW{\mathbf{W}}
\def\vX{\mathbf{X}}
\def\vY{\mathbf{Y}}
\def\vZ{\mathbf{Z}}

\def\va{\mathbf{a}}
\def\vb{\mathbf{b}}
\def\vc{\mathbf{c}}
\def\vd{\mathbf{d}}
\def\ve{\mathbf{e}}
\def\vf{\mathbf{f}}
\def\vg{\mathbf{g}}
\def\vh{\mathbf{h}}
\def\vi{\mathbf{i}}
\def\vj{\mathbf{j}}
\def\vk{\mathbf{k}}
\def\vl{\mathbf{l}}
\def\vm{\mathbf{m}}
\def\vn{\mathbf{n}}
\def\vo{\mathbf{o}}
\def\vp{\mathbf{p}}
\def\vq{\mathbf{q}}
\def\vr{\mathbf{r}}
\def\vs{\mathbf{s}}
\def\vt{\mathbf{t}}
\def\vu{\mathbf{u}}
\def\vv{\mathbf{v}}\def\tvv{\tilde{\mathbf{v}}}
\def\vw{\mathbf{w}}
\def\vx{\mathbf{x}}\def\tvx{\tilde{\mathbf{x}}}
\def\vy{\mathbf{y}}
\def\vz{\mathbf{z}}

%%% --Numerical analysis related
%\newcommand{\nxt}{^{n+1}}
%\newcommand{\pvs}{^{n-1}}
%\newcommand{\hfnxt}{^{n+\frac12}}

%%%%%%%%%%%%%%%%%%%%%%%%%%%%%%%%%%%%%%%%%%%%%%%%%%%%%%%%%%%%%%%%%%%%%%%%%%%%%%%%%%%%%%%%%%%%%%%%%%%%%%%%%%%%%%%%%%%%%%%%%%%%%%%%%%%%%%%%%%%%%%%%%%%%%%%%%%%%%%%%%%%%%%%%%%%%%%%%%%%%%%%%%%%%%%%%%%%%%%
\begin{document}
\subsubsection*{Problem 1. \textit{Barzilai-Borwein step sizes.}}
Consider that $s_k = x_k - x_{k-1}$ and $y_k = \nabla f(x_k) - \nabla f(x_{k-1})$, the Lipschitz continuity of $\nabla f$ implies
\begin{equation*}
  \|y_k\|_\ast \leq L\|s_k\|.
\end{equation*}
Cauchy inequality then yields $s_k^T y_k \leq L\|s_k\|^2$ and $\|y_k\|_\ast^2 \leq s_k^T y_k$, from which we conclude
\begin{equation*}
  \/{\|s_k\|^2}{s_k^Ty_k} \geq \/1L, \qquad \/{s_k^Ty_k}{\|y_k\|_\ast^2} \geq \/1L
\end{equation*}

\subsubsection*{Problem 2.}
First note that $F(x) = Ax + b$ gives $F(x) - F(y) = A(x-y)$ regardless of symmetry of the matrix $A$. Also any matrix can be decompose into symmetric and skew-symmetric parts, i.e.
\begin{equation*}
  A = \l(\/{A+A^T}{2}\r) + \l(\/{A-A^T}{2}\r) = \bar A + A_c.
\end{equation*}
(a) The condition is $(F(x)-F(y))^T(x-y) = (x-y)^TA^T(x-y) \geq 0$. (1) For symmetric $A$, this means $A$ is semi-positive definite. (2) For skew-symmetric $A$, $v^TAv = (v^TAv)^T = v^TA^Tv = -v^TAv = 0$ and the condition will always hold true. (3) Using the decomposition of symmetric and skew-symmetric parts, a general non-symmetric matrix satisfying the condition needs to have semi-positive definite symmetric part. \\
\\
(b) Similar to part (a), the condition $(x-y)^TA^T(x-y) > 0$ means that (1) the symmetric matrix is positive-definite, (2) skew-symmetric matrices cannot satisfy this condition, or concluding these two cases, (3) any general matrix needs to be symmetric positive definite in order to satisfy this condition. \\
\\
(c) Similar to above, the condition $(x-y)^T A^T (x-y) > m\|x-y\|_2^2$ means that (1) the symmetric matrix is positive-definite with eigenvalues all greater than or equal to $m$\footnote{This is true since we are working with finite dimensional linear transformations. We can always take the $n$ eigenvalues of $A$ and take the minimum. A symmetric positive definite matrix can have only positive eigenvalues thus the minimum $m$ will be a positive constant. }, (2) skew-symmetric matrices cannot satisfy this condition, or as a result (3) any general matrix needs to be symmetric positive definite with eigenvalues all greater than or equal to $m$ in order to satisfy this condition. \\
\\
(d) The condition is $\|F(x)-F(y)\|_2 = \|A(x-y)\|_2 \leq L\|x-y\|_2$, or, equivalently, $\|A\|_L \leq L$ where $\|A\|_L = \sup_{x\neq 0} \/{\|Ax\|}{\|x\|}$ is the operator norm. The operator norm is equal to (1) the largest eigenvalue for symmetric matrices or (3) the largest singular value for general matrices. Particularly for (2) skew-symmetric matrices, since they have only purely imaginary eigenvalues, the condition means that the absolute value of the eigenvalues must be all less than or equal to $L$. \\
\\
(e) The condition is $(x-y)^TA^T(x-y) \geq \/1L \|A(x-y)\|_2^2 = \/1L (x-y)^T A^T A (x-y)$. (1) For symmetric $A$, this means that $A$ is semi-positive definite and the largest eigenvalue of $A$ is less than or equal to $L$. (2) For skew-symmetric $A$, left-hand-side is always $0$ (as shown in part (a)), yet as long as $A \neq 0$, we can find a vector $v = x-y$ such that $\|Av\| >0$ and the condition simply cannot hold. (3) For general matrices, not much simplification can be done; we conclude it with the following criterion:
\begin{equation*}
  \/12 (A + A^T)  - \/1L A^TA\succeq 0.
\end{equation*}
Note that this matrix is symmetric indeed and the condition is equivalent to this matrix being semi-positive definite.

% I tried using singular-value, Schur's, polar, and symmetric-skew-symmetric decomposition and still have yet to find a satisfactory simplification. Here's some scribble:
% \begin{itemize}
%   \item If $A = U\Sigma V^T$: singular value decomposition\\
%     then $v^TAv = (U^T v)^T \Sigma V^T v$ and $\|Av\|_2^2 = (V^T v)^T \Sigma^2 V^T v$ \\
%     Although $U, V$ preserves norm and this seems to imply relation between the largest entry of $\Sigma$ and $L$, the asymmetry of $U^T v$ and $V^T v$ makes it impossible to conclude anything.  \\
%     One observation related to Cauchy inequality: \\
%     $ (U^T v)^T (\Sigma V^T v) \leq \|U^T v\|_2 \|\Sigma V^T v\|_2 $ \\
%     But this is not of much use since the inequality isn't of the desired direction.

%   \item If $A = QUQ^\ast$: Schur's decomposition ($Q$: unitary, $U$: upper triangular) \\
%     then $v^TAv = (Q^\ast v)^T U Q^\ast v$ and $\|Av\|_2^2 = (Q^\ast v)^T U^T U Q^\ast v$ \\
%     However $U^TU$ destroys the upper triangular advantage of $U$.

%   \item If $A = RS$: polar decomposition ($R$: unitary, $S$: positive definite) \\
%     then $v^TAv = (R^T v)^T S v$ and $\|Av\|_2^2 = v^T S^2 v$ \\
%     This doesn't really give anything more than singular value decomposition and the asymmetry of $R^T v$ and $v$ makes it impossible to conclude.

%   \item If $A = \bar A + A_c$: symmetric-skew-symmetric decomposition ($\bar A$: symmetric, $A_c$: skew-symmetric) \\
%     then $A^T A = \bar A^2 - A_c^2 + \bar A A_c - A_c \bar A$ \\
%     However since $\bar A$ doesn't necessarily commute with $A_c$ (it depends on if $A$ and $A^T$ commute), this decomposition doesn't provide much insight.
% \end{itemize}
\end{document}



