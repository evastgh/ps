\documentclass[12pt,a4paper]{article}
	%[fleqn] %%% --to make all equation left-algned--

% \usepackage[utf8]{inputenc}
% \DeclareUnicodeCharacter{1D12A}{\doublesharp}
% \DeclareUnicodeCharacter{2693}{\anchor}
% \usepackage{dingbat}
% \DeclareRobustCommand\dash\unskip\nobreak\thinspace{\textemdash\allowbreak\thinspace\ignorespaces}
\usepackage[top=1.1in, bottom=.9in, left=.9in, right=.9in]{geometry}
%\usepackage{fullpage}

\usepackage{fancyhdr}\pagestyle{fancy}\rhead{June 13, 2019}\lhead{Math 33A -- Final}

\usepackage{nicefrac, tabularx}

\usepackage{amsmath,amssymb,amsthm,amsfonts,microtype,stmaryrd}
	%{mathtools,wasysym,yhmath}

\usepackage[usenames,dvipsnames]{xcolor}
\newcommand{\blue}[1]{\textcolor{blue}{#1}}
\newcommand{\red}[1]{\textcolor{red}{#1}}
\newcommand{\gray}[1]{\textcolor{gray}{#1}}
\newcommand{\fgreen}[1]{\textcolor{ForestGreen}{#1}}

\usepackage{mdframed}
	%\newtheorem{mdexample}{Example}
	\definecolor{warmgreen}{rgb}{0.8,0.9,0.85}
	% --Example:
	% \begin{center}
	% \begin{minipage}{0.7\textwidth}
	% \begin{mdframed}[backgroundcolor=warmgreen, 
	% skipabove=4pt,skipbelow=4pt,hidealllines=true, 
	% topline=false,leftline=false,middlelinewidth=10pt, 
	% roundcorner=10pt] 
	%%%% --CONTENTS-- %%%%
	% \end{mdframed}\end{minipage}\end{center}	

%\usepackage{graphicx} \graphicspath{ {/path/} }
	% --Example:
	% \includegraphics[scale=0.5]{picture name}
%\usepackage{caption} %%% --some awful package to make caption...

%\usepackage{hyperref}\hypersetup{linktocpage,colorlinks}\hypersetup{citecolor=black,filecolor=black,linkcolor=black,urlcolor=black}

%%% --Text Fonts
%\usepackage{times} %%% --Times New Roman for LaTeX
%\usepackage{fontspec}\setmainfont{Times New Roman} %%% --Times New Roman; XeLaTeX only

%%% --Math Fonts
\renewcommand{\v}[1]{\ifmmode\mathbf{#1}\fi}
%\renewcommand{\mbf}[1]{\mathbf{#1}} %%% --vector
%\newcommand{\ca}[1]{\mathcal{#1}} %%% --"bigO"
%\newcommand{\bb}[1]{\mathbb{#1}} %%% --"Natural, Real numbers"
%\newcommand{\rom}[1]{\romannumeral{#1}} %%% --Roman numbers

%%% --Quick Arrows
\newcommand{\ra}[1]{\ifnum #1=1\rightarrow\fi\ifnum #1=2\Rightarrow\fi\ifnum #1=3\Rrightarrow\fi\ifnum #1=4\rightrightarrows\fi\ifnum #1=5\rightleftarrows\fi\ifnum #1=6\mapsto\fi\ifnum #1=7\iffalse\fi\fi\ifnum #1=8\twoheadrightarrow\fi\ifnum #1=9\rightharpoonup\fi\ifnum #1=0\rightharpoondown\fi}

%\newcommand{\la}[1]{\ifnum #1=1\leftarrow\fi\ifnum #1=2\Leftarrow\fi\ifnum #1=3\Lleftarrow\fi\ifnum #1=4\leftleftarrows\fi\ifnum #1=5\rightleftarrows\fi\ifnum #1=6\mapsfrom\ifnum #1=7\iffalse\fi\fi\ifnum #1=8\twoheadleftarrow\fi\ifnum #1=9\leftharpoonup\fi\ifnum #1=0\leftharpoondown\fi}

%\newcommand{\ua}[1]{\ifnum #1=1\uparrow\fi\ifnum #1=2\Uparrow\fi}
%\newcommand{\da}[1]{\ifnum #1=1\downarrow\fi\ifnum #1=2\Downarrow\fi}

%%% --Special Editor Config
\renewcommand{\ni}{\noindent}
\newcommand{\onum}[1]{\raisebox{.5pt}{\textcircled{\raisebox{-1pt} {#1}}}}

\newcommand{\claim}[1]{\underline{``{#1}":}}

\renewcommand{\l}{\left}
\renewcommand{\r}{\right}

%\newcommand{\casebrak}[2]{\left \{ \begin{array}{l} {#1}\\{#2} \end{array} \right.}
\newcommand{\ttm}[4]{\l[\begin{array}{cc}{#1}&{#2}\\{#3}&{#4}\end{array}\r]} %two-by-two-matrix
\newcommand{\tv}[2]{\l[\begin{array}{c}{#1}\\{#2}\end{array}\r]}

\def\dps{\displaystyle}

\let\italiccorrection=\/
\def\/{\ifmmode\expandafter\frac\else\italiccorrection\fi}


%%% --General Math Symbols
\def\bc{\because}
\def\tf{\therefore}

%%% --Frequently used OPERATORS shorthand
\newcommand{\INT}[2]{\int_{#1}^{#2}}
% \newcommand{\UPINT}{\bar\int}
% \newcommand{\UPINTRd}{\overline{\int_{\bb R ^d}}}
\newcommand{\SUM}[2]{\sum\limits_{#1}^{#2}}
\newcommand{\PROD}[2]{\prod\limits_{#1}^{#2}}
% \newcommand{\CUP}[2]{\bigcup\limits_{#1}^{#2}}
% \newcommand{\CAP}[2]{\bigcap\limits_{#1}^{#2}}
% \newcommand{\SUP}[1]{\sup\limits_{#1}}
% \newcommand{\INF}[1]{\inf\limits_{#1}}
\DeclareMathOperator*{\argmin}{arg\,min}
\DeclareMathOperator*{\argmax}{arg\,max}
\newcommand{\pd}[2]{\frac{\partial{#1}}{\partial{#2}}}
\def\tr{\text{tr}}

\renewcommand{\o}{\circ}
\newcommand{\x}{\times}
\newcommand{\ox}{\otimes}

%%% --Frequently used VARIABLES shorthand
\def\R{\ifmmode\mathbb R\fi}
\def\N{\ifmmode\mathbb N\fi}
\renewcommand{\O}{\mathcal{O}}

%%%%%%%%%%%%%%%%%%%%%%%%%%%%%%%%%%%%
\begin{document}
\noindent\textbf{Instructions:}
\begin{itemize}
    \item Follow directions and answer questions with requested supporting work.
    \item Clearly indicate your answer in the allotted space or by putting a box around it.
    \item No cellphones, laptops, books, notes, supporting materials, or external aids are allowed on this exam.
\end{itemize}

\vskip 30pt
\begin{center}
{\renewcommand{\arraystretch}{1.3}
\begin{tabular}{ rl }
    Name: & \underline{\hskip 150pt} \\
	  % & \\
    UID:  & \underline{\hskip 150pt} \\
	  % & \\
    % Date: & \underline{\hskip 150pt}
\end{tabular}
}

\vskip 30pt

{\renewcommand{\arraystretch}{1.2}
\begin{tabular}{|c|c|}
    \hline
    Problem \# & \;\;\;\;\;Score\;\;\;\;\; \\
    \hline
    1 & \\
    \hline
    2 & \\
    \hline
    3 & \\
    \hline
    4 & \\
    \hline
    5 & \\
    \hline
    6 & \\
    \hline
    7 & \\
    \hline
    8 & \\
    \hline
    9 & \\
    \hline
    % 10 & \\
    % \hline
    Total & \\
    \hline
\end{tabular}
}
\end{center}

\newpage
\noindent1. True or False. \textbf{(2 points each)}\\
(a) A symmetric matrix must be orthogonal. \\
\gray{False}
\\
\\
\\
(b) Orthogonal mappings preserves the length of vectors but not the angle between vectors. \\
\gray{False}
\\
\\
\\
(c) An upper triangular matrix must be diagonalizable. \\
\gray{False}
\\
\\
\\
(d) An eigenbasis is a basis of an eigenspace. \\
\gray{False}
\\
\\
\\
(e) An orthogonal matrix must be invertible. \\
\gray{True}
\\
\\
\\
(f) If two matrices are similar, they must have the same characteristic polynomials. \\
\gray{True}
\\
\\
\\
(g) Let $A\in \R^{n\x n}$ be a square matrix. If all eigenvalues of $A$ are zero, then $A$ must be the zero matrix.\\
\gray{False}
\\
\\
\\
(h) Let $A \in \R^{m\x n}$ be a tall and thin matrix ($m > n$). If the columns of $A$ are linearly independent then that $A^TA$ is invertible. \\
\gray{True}
\\
\\
\\
(i) The matrices $A$ and $A^T$ share the same set of eigenvalues. \\
\gray{True}
\\
\\
\\
(j) The matrices $A$ and $A^{-1}$ share the same set of eigenvectors. \\
\gray{True}
\\
\\
\\


\newpage
\noindent2. Answer the following questions. \textbf{(2 points each)}\\
(a) What is the definition of a subspace on $\R^n$? \\
\gray{Contains zero and is closed under linear combination.}
\\
\\
\\
\\
\\
\\
(b) Let $A\in \R^{n\x n}$ be a square matrix. Write down at least two conditions that are equivalent to $A$ being invertible; write down more conditions to earn up to two bonus points. \\
\gray{(independent columns, nonzero determinant, subjective, injective, exist a matrix inverse, linear system has unique solutions, etc)}
\\
\\
\\
\\
\\
\\
(c) Let $A = \ttm abcd \in \R^{2\x 2}$. Consider the triangle with vertices $(0, 0)$, $(a,c)$, and $(b, d)$. What is the area of this triangle? \\
\gray{$\/12 |\det(A)|$.} 
\\
\\
\\
\\
\\
\\
(d) State the spectral theorem. \\
\gray{A matrix is symmetric if and only if it is orthogonally diagonalizable.}
\\
\\
\\
\\
\\
\\
(e) Write down the definition of an orthogonal matrix. \\
\gray{$Q$ is orthogonal if its columns are orthonormal or if $Q^TQ = I$.\\}
\\
\\
\\
\\
\\
\\
(f) State the rank-nullity theorem. \\
\gray{The sum of rank and nullity of a matrix $A$ is the same as the width of $A$. \\}

\newpage
\noindent3. Consider the following vectors
\[
v_1 = 
\l[
\begin{array}{c}
    1\\2\\3
\end{array}
\r]
, \;
v_2 = 
\l[
\begin{array}{c}
    0\\1\\2
\end{array}
\r]
, \;
v_3 = 
\l[
\begin{array}{c}
    0\\0\\1
\end{array}
\r].
\]
(a) \textbf{(2 points) }Show that $\mathcal B = \{v_1, v_2, v_3\}$ is a basis for $\R^3$.\\
\gray{The matrix $P = [v_1, v_2, v_3]$ is an lower triangular matrix with nonzero diagonals; it is invertible. Hence its columns are linear independent, and they form a basis. \\}
\\
(b) \textbf{(2 points) }Find the change of coordinate matrix $S$ such that $S[x]_{\mathcal E} = [x]_{\mathcal B}$ for all $x \in \R^3$; here $\mathcal E$ denotes the standard basis. \\
\gray{$S = P^{-1}$. \\}
\\
(c) \textbf{(2 points) }Find the coordinates $[x]_{\mathcal B}$ for the vector $x = \l[
\begin{array}{c}
    1\\0\\1
\end{array}
\r].$ \\
\gray{$[x]_{\mathcal B} = Sx$. \\}
\\
(d) \textbf{(4 points) }Find the matrix $B = [T]_{\mathcal B}$ of the linear transformation $T(x) = Ax$ with respect to basis $\mathcal B$ where 
\[
    A = \l[
\begin{array}{ccc}
    -1 & 1 & 0 \\
    0 & -2 & 2 \\
    3 & -9 & 6
\end{array}
\r].
\]
\gray{$B = SAS^{-1}.$\\}

% \newpage
% \noindent4. Consider two nonzero vectors $u = (u_1, u_2, \cdots, u_n), v = (v_1, v_2, \cdots, v_n) \in \R^n$. Define
% $$A = uv^T = \l[
% \begin{array}{cccc}
%     u_1v_1 & u_1v_2 & \cdots & u_1v_n\\
%     u_2v_1 & u_2v_2 & & \\
%     \vdots & & \ddots & \vdots\\
%     u_nv_1 & & \cdots & u_nv_n
% \end{array}
% \r]
% $$
% (a) Show that $im(A) = \{\alpha u : \alpha \in \R\}$ and $rank(A) = 1$.\\
% \\
% (b) Show that $\dim(ker(A)) = n-1$. \\
% \\
% (c) Suppose $u = |u|\bar u, v = |v|\bar v$ ($\bar u, \bar v$ are the unit vectors from normalizing $u, v$). Assume further that $\bar v, x_2, \cdots, x_n$ form an orthonormal basis of $\R^n$. Show that $x_2, \cdots, x_n$ form an orthonormal basis of $ker(A)$. \\
% \\
% (d) Assuming $u, v$ are linearly independent. Find an eigenbases and diagonalize $A$. \\
% \\
% (e) \textbf{(Bonus)} On the other hand, if $u, v$ are linearly dependent, is $A$ still diagonalizable?

\newpage
\noindent4. Let $w = (1, 2, -2) \in \R^3$ and $W = span(w)$. \\
(a) \textbf{(2 points) }Find the matrix $A$ of the orthogonal projection onto $W$. \\
\\
(b) \textbf{(2 points) }What is $rank(A)$? \\
\\
(c) \textbf{(2 points) }Find a basis for $ker(A)$. \\
\\
(d) \textbf{(2 points) }What is the set of eigenvalues of $A$? \\
\\
(e) \textbf{(2 points) }Find a basis $\mathcal B$ such that $[A]_{\mathcal B}$ is diagonal. \\
\\
\gray{
    (a) Let $\bar w = \frac{w}{|w|} = \frac13 w$. $A = \bar w\bar w^T$. \\
    (b) 1. \\
    (c) $ker(A) = W^\perp = span(v_1, v_2)$ where $v_1 = (0, 1, 1) v_2 = (2, -1, 0)$. Note: answer to this one is not unique. \\
    (d) $\{0, 1\}$. \\
    (e) $\mathcal B = \{w, v_1, v_2\}$ (also not unique).
}

\newpage
\noindent5. Consider the subspace $W$ of $\R^4$ spanned by the vectors 
$$
v_1 = 
\l[
\begin{array}{c}
    1\\1\\1\\1
\end{array}
\r]
, \;
v_2 = 
\l[
\begin{array}{c}
    1\\9\\-5\\3
\end{array}
\r] .
$$
(a) \textbf{(3 points) }Perform Gram-Schmidt process and find an orthonormal basis $u_1, u_2$ for $W$. \\
\\
(b) \textbf{(3 points) }Find the QR-factorization of the matrix $A = [v_1, v_2] \in \R^{4\x 2}$. \\
\gray{$u_1 = \/12v_1, v_2^\perp = v_2 - u_1^Tv_2u_1 = v_2 - 4u_1 = (-1, 7, -7, 1), u_2 = \/1{10}v_2^\perp, Q = [u_1, u_2], R = \ttm240{10}.$\\}
\\
(c) \textbf{(2 points) }Write down the matrix $B$ of the orthogonal projection onto $W$ in terms of $Q$ and $R$. \\
\gray{$B = QQ^T = u_1u_1^T + u_2u_2^T.$\\}
\\
(d) \textbf{(2 points) }Let $b \in \R^4$. Write down the least-squares solution to $Ax = b$ in terms of $Q$, $R$, and $b$. \\
\gray{The solution is $x = R^{-1}Q^Tb$. \\}


\newpage
\noindent6. Consider the following four vectors in $\R^4$.
$$
v_1 = 
\l[
\begin{array}{c}
    1\\ 0\\ 0\\ 0
\end{array}
\r]
, \;
v_2 = 
\l[
\begin{array}{c}
    0\\1\\0\\2
\end{array}
\r]
, \;
v_3 = 
\l[
\begin{array}{c}
    0\\0\\1\\0
\end{array}
\r]
, \;
v_4 = 
\l[
\begin{array}{c}
    0\\1\\0\\1
\end{array}
\r].
$$
(a) \textbf{(3 points) }Find the 4-volume of the 4-parallelepiped defined by $v_1, v_2, v_3, v_4$.\\
\gray{$A = [v_1, v_2, v_3, v_4], |\det(A)| = |-1| = 1.$\\}
\\
(b) \textbf{(3 points) }Find the 3-volume of the 3-parallelepiped defined by $v_2, v_3, v_4$. \\
\gray{Since the first row of $A$ is $1, 0, 0, 0$, the answer is 1. \\}
\\
(c) \textbf{(4 points) }Find the 2-volume of the parallelogram defined by $v_3, v_4$. \\
\gray{$A = [v_3, v_4], \sqrt{\det(AA^T)} = \sqrt{\det\ttm1002} = \sqrt2.$\\}






\newpage
\noindent7. Consider a matrix $A \in \R^{3\x 3}$ given by 
$$
A = 
\l[
\begin{array}{rrr}
    1 & 1 & 0\\
    0 & 2 & 2\\
    0 & 0 & 3
\end{array}
\r]
. 
$$
(a) \textbf{(5 points) }Find all (real) eigenvalues, a basis of each eigenspace, and diagonalize $A$ if possible. \\
\gray{The eigen-decomposition $A = SDS^{-1}$ is given by 
    $$
S = 
\l[
\begin{array}{rrr}
    1&1&1\\
    0&1&2\\
    0&0&1
\end{array}
\r], \quad
D = 
\l[
\begin{array}{rrr}
    1&&\\
     &2&\\
     &&3
\end{array}
\r], \quad
S^{-1} = 
\l[
\begin{array}{rrr}
    1&-1&1\\
    0&1&-2\\
    0&0&1
\end{array}
\r].
    $$
}
\\
(b) \textbf{(5 points) }Do the same for $A^{-1}$. \\
\gray{$A^{-1} = SD^{-1}S^{-1}$.\\}
\\

% \newpage
% \noindent8. Consider the dynamical system
% \begin{align*}
%     x_1(t+1) &= 0.5 x_1(t) + 0.25 x_2(t) \\
%     x_2(t+1) &= 0.5 x_1(t) + 0.75 x_2(t),
% \end{align*}
% with an initial state $x(0) = (0.4, 0.6)$. \\
% \\
% (a) Find a matrix $A \in \R^{2\x 2}$ such that $x(t) = A^tx(0)$. \\
% \gray{
% $$A = 
% \l[
% \begin{array}{rr}
%     0.5 & 0.25 \\
%     0.5 & 0.75
% \end{array}
% \r]. 
% $$
% }
% \\
% (b) Find a closed formula for $x(t)$, expressing the vector $A^tx(0)$ as a function of $t$. \\
% \gray{
%     Eigen-decomposition of $A = SDS^{-1}$,
% $$S = 
% \l[
% \begin{array}{rr}
%     -1 & 0.5 \\
%     1 & 1 
% \end{array}
% \r], \quad 
% D = 
% \l[
% \begin{array}{rr}
%     0.25 & 0 \\
%     0 & 1
% \end{array}
% \r], \quad
% S^{-1} = 
% \l[
% \begin{array}{rr}
%     \frac{-2}3 & \frac13 \\
%     \frac23 & \frac23
% \end{array}
% \r]
% .
% $$
% $x(t) = A^tx(0) = SD^tS^{-1}x(0)$. \\
% }
% \\
% (c) What happens in the long run? Find $\lim_{t\ra1 \infty} A^t x(0)$ if it exists. \\
% \gray{$[x(0)]_\beta = S^{-1}x(0) = (\nicefrac{-2}{30}, \nicefrac23)$ and the limit is the equilibrium eigenvector $(0.5, 1)$ times $2/3$. \\ }


\newpage
\noindent8. Consider the quadratic form
$$q(x_1, x_2, x_3) = 4x_1x_2 + 4x_1x_3 + x_2^2 - x_3^2.$$
Answer the following questions. To prevent calculation error in part (a) or (b) costing you points for the entire problem, consider answering part (c) and (d) not only specifically to this problem but also for general results.\\
\\
(a) \textbf{(2 points)} Find the symmetric matrix $A$ such that $q(x) = x^TAx$ for $x \in \R^3$. \\
\\
(b) \textbf{(5 points)} Find the orthogonal eigenvalue decomposition $A = SDS^T$ where $S$ is an orthogonal matrix and $D$ is diagonal. \\
\\
(c) \textbf{(3 points)} Determine the definiteness of the quadratic form $q$. \\ 
\\
(d) \textbf{(Bonus: 2 points)} What is the maximal value $q$ takes on the set of all unit vectors? i.e. what is $\max_{\|x\| = 1} q(x)$?\\
\\
\gray{This is quiz 10 problem yo. Look up Tim's solutions!}


\newpage
\noindent9. Consider the 2 by 3 matrix
$$A = 
\l[
\begin{array}{ccc}
    0 & 1 & 1\\
    1 & 1 & 0\\
\end{array}
\r].
$$
(a) \textbf{(3 points)} Find the singular values of $A$. \\
\gray{$$
    A^TA = 
\l[
\begin{array}{ccc}
    1 & 1 & 0\\
    1 & 2 & 1\\
    0 & 1 & 1
\end{array}
\r].$$
Eigenvalues of $A^TA$ are 3, 1, 0. Singular values of $A$ are $\sqrt3, 1$ (not including $0$).}
\\
(b) \textbf{(4 points)} Write down the process of finding the singular value decomposition
$$A = U\Sigma V^T,$$
where $U$ is an 2 by 2 orthogonal matrix, $\Sigma$ is a 2 by 3 diagonal matrix with all nonnegative diagonal entries, and $V$ is an 3 by 3 orthogonal matrix. \\
\gray{Compute the diagonal orthogonalization of $A^TA = SDS^T$. Let $V = S$. Let $\sigma_i = \sqrt{d_{ii}}$ and fill the diagonal of $\Sigma$, which is of the same size as $A$. Compute $U$ according to $Av_i = \sigma_i u_i$. Use Gram-Schmidt to find the orthonormal basis to fill the $u_i$'s with $\sigma_i = 0$.}
\\
(c) \textbf{(3 points)} Compute the singular value decomposition for $A$. \\
\gray{The eigenvectors of $A^TA$ are $v_1 = \/1{\sqrt6}(1, 2, 1), v_2 = \/1{\sqrt2}(1, 0, -1), v_3 = \/1{\sqrt3}(1, -1, 1)$; solve $Av_1 = \/1{\sqrt6}(3, 3) = \sqrt3 u_1, Av_2 = \/1{\sqrt2}(-1, 1)u_2, Av_3 = (0, 0)$ for $u_1, u_2$. Note that these vectors are unique up to a sign flip.}
\\





% \newpage
% \noindent10.
% (a) Consider a quadratic form
% $$q(x) = x^T Ax,$$
% where $A$ is a symmetric $n \x n$ matrix. Let $v$ be a unit eigenvector of $A$, with associated eigenvalue $\lambda \in \R$. Find $q(v)$. \\
% \gray{$q(v) = \lambda$. \\} 
% \\
% (b) Determine for which value of $a, b, c \in \R$ can the matrix
% $$B = 
% \l[
% \begin{array}{rr}
%     a& b \\
%     b & c
% \end{array}
% \r]
% $$
% be positive definite. \\
% \gray{$a > 0$ and $ac-b^2 > 0$. \\} 
% \\
% (c) Suppose matrix $C \in \R^{2 \x 2}$ has singular value $\sigma_1 = 1, \sigma_2 = 0.1$. Show that for all unit vector $v$ ($|v| = 1$), 
% $$0.1 \leq |Av| \leq 1.$$
% \gray{$A = U\Sigma V^T, \Sigma = diag(1, 0.1)$. For $\bar v = V^Tv$, which is also an unit vector ($|V^T v| = |v| = 1$), $Av = U\Sigma \bar v$ and $|Av| = |U\Sigma \bar v| = |\Sigma \bar v| = \sqrt{1\cdot \bar v_1^2 + 0.1\cdot\bar v_2^2} = \sqrt{1-0.9\bar v_2^2} \in [0.1, 1]$.}


\end{document}

